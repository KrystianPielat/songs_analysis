\chapter{Conclusion}
\label{cha:conclusion}
%---------------------------------------------------------------------------

\section{Summary of Contributions}
\label{sec:summaryofcontributions}

This thesis has provided an extensive analysis of the relationships between
acoustic parameters, lyrical content, and metadata in characterizing music
tracks. By integrating well organized data collection methodology, advanced
feature engineering, explainable AI methods, machine learning models and
statistical methods, this study accomplished the following:

\begin{itemize}
  \item Constructed a diverse dataset by combining Spotify metadata and audio
    features, textual features derived from lyrics, and acoustic features
    extracted from downloaded MP3 files, representing over 3,500 songs.

  \item Conducted an in-depth EDA to identify trends, patterns, and distinctive
    characteristics within acoustic and lyrical features across genres,
    leveraging advanced statistical methods.

  \item Characterized distinctive traits of the genres included in the study,
    providing insights into their unique acoustic and lyrical attributes. 

  \item Trained machine learning models for various predictive tasks, including
    popularity, explicitness, sentiment, and genre classification, and
    identified key features influencing these attributes. The models were
    tested against baseline models to evaluate their performance,
    and explainable AI methodologies such as SHAP were used to enhance
    interpretability and investigate feature importance and interactions.

  \item Applied Latent Dirichlet Allocation (LDA) to identify and analyze
    thematic patterns in lyrics, offering insights into the recurring themes
    and their distribution across genres and time periods, while exploring
    their distinct characteristics.

  \item Investigated temporal trends in music characteristics, offering
    insights into the evolution of music styles and listener preferences.

  \item Established a reusable framework for data collection, feature
    extraction, and XAI-driven analysis, providing foundation for further
    research in music analysis.
\end{itemize}






%---------------------------------------------------------------------------
\section{Limitations of the Study}
\label{sec:limitations}

This study faced several key limitations that should be considered when
interpreting the results:

\subsubsection*{Selection Bias}

The dataset was created through stratified sampling of songs on Spotify, based
on genre and release decade. However, this stratification relied on Spotify's
search algorithm, introducing a potential bias influenced by the algorithm's
underlying logic and priorities. Factors such as the musical preferences of the
user querying the data, the time of year, and the specific version of the
search algorithm at the time of data collection could all have impacted the
selection of songs, potentially skewing the dataset toward certain trends or
preferences. Collection of truly random sample for the purpose of this study
would have posed significant challenges, explained in \textit{Sampling
Method} (\ref{sec:samplingmethod}).




\subsubsection*{Dimensionality}

The dataset was relatively small due to computational and time constraints.
This limitation restricted the complexity of analyses and the generalizability
of insights. Larger datasets could have enhanced the performance of the models
and provided more robust findings.

\subsubsection*{Dependency on Spotify}
Many features used in this study were derived from Spotify's metadata and
algorithms. For instance, genre assignments, popularity metrics, and features
such as danceability and loudness depend on Spotify's estimations, which may
not always align with objective or universal measures.



%---------------------------------------------------------------------------
\section{Recommendations for Future Work}
\label{sec:recommendationsforfuturework}


\subsubsection*{Expand Dataset Size and Diversity}
Collect a larger and more diverse dataset. More data would not only improve the
quality and robustness of trained models but also enable more comprehensive and
accurate analyses, uncovering complex relationships and trends in the
data.

\subsubsection*{Reduce Dependency on Spotify}
Incorporate alternative data sources to minimize reliance on Spotify's metadata
and algorithmically generated features. These could include publicly available
audio datasets, other music streaming platforms, or custom feature extraction
from raw audio files.


\subsubsection*{Improve Feature Engineering}
Enhance the feature engineering process, with a particular focus on extracting
more nuanced and comprehensive features from MP3 files. Improved feature
engineering would provide a deeper understanding of music's structure and
characteristics, contributing to more accurate and insightful analyses.

\subsubsection*{Utilize Advanced Modeling Techniques}
Experiment with a variety of models, including more complex architectures, to
better capture the relationships in the data. Broaden the hyperparameter search
in Optuna to optimize model performance and identify the best configurations.


\subsubsection*{Refine Clustering and Topic Modeling}
Extend the use of clustering and topic modeling techniques to uncover more
nuanced relationships between genres and lyrical themes.
