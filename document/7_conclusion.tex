\chapter{Conclusion}
\label{cha:conclusion}
%---------------------------------------------------------------------------

\section{Summary of Contributions}
\label{sec:summaryofcontributions}

This thesis has provided an extensive analysis of the relationships between
acoustic parameters, lyrical content, and metadata in characterizing music
tracks. By integrating well organized data collection methodology, advanced
feature engineering, explainable AI methods, machine learning models and
statistical methods, this study accomplished the following:

\begin{itemize}
  \item Constructed a diverse dataset by combining Spotify metadata and audio
    features, textual features derived from lyrics, and acoustic features
    extracted from downloaded MP3 files, representing over 3,500 songs;

  \item Employed advanced statistical methods to conduct extensive exploratory
    data analysis to identify relationships, patterns, and distinctive
    characteristics within acoustic and lyrical features across genres;

  \item Found distinctive traits of the genres included in the study,
    describing their unique acoustic and lyrical attributes;

  \item Trained machine learning models for various predictive tasks, including
    regression of popularity and  classification of explicitness, sentiment,
    and genre. and identified key features influencing these attributes. The
    models were tested against baseline models to evaluate their performance,
    and explainable AI methodologies such as SHAP were used to enhance
    interpretability and investigate feature importance and interactions;

  \item Applied Latent Dirichlet Allocation (LDA) to identify and analyze
    thematic patterns in lyrics, providing insights into the recurring themes
    that can be found in the songs included in the dataset as well as their
    distribution across genres and decades, and their distinct characteristics;

  \item With the use of statistical methods it investigated temporal trends in
    music characteristics, identifying changes reflecting the evolution of
    music styles and listener preferences across the decades;

  \item Created a reusable framework for data collection, feature engineering
    and XAI-driven analysis for music, providing foundation for further
    research in similar domains.
\end{itemize}






%---------------------------------------------------------------------------
\section{Limitations of the Study}
\label{sec:limitations}

When interpreting the results, several limitations of the  study should be
considered:

\subsubsection*{Selection Bias}

In order to acquire the data in the data collection phase the method chosen for
sampling the songs available on Spotify utilized  stratified sampling based on
genre and release decade. This method of sampling relied on Spotify's search
algorithm, and therefore introduced a potential bias influenced by factors that
this thesis did not account for. Factors like the musical preferences of the
user querying the data, the time of the year at which the data was acquired,
and many other features influencing the search algorithm's results likely
impacted the selection of songs, potentially skewing the dataset. Collection of
truly random sample for the purpose of this study would have posed significant
challenges, explained in \textit{Sampling Method} (\ref{sec:samplingmethod}).


\subsubsection*{Dimensionality}

Due to computational and memory constraints the dataset was relatively small.
A larger sample would most likely provide better quality insights and improve
the performance of the predictive models, leading to more robust findings. 


\subsubsection*{Dependency on Spotify}
Many features used in this study were derived from Spotify's metadata and
algorithms. Genre labels, popularity metrics, and features such as danceability
and loudness depend on Spotify's estimations, which may not always align with
objective or universal understanding of those properties.



%---------------------------------------------------------------------------
\section{Recommendations for Future Work}
\label{sec:recommendationsforfuturework}


\subsubsection*{Gather More Data and Increase Diversity}
Collecting more data with possibly a different sampling method and adding
more diverse features would provide a better foundation for research. It would
not only improve the quality of trained predictive models but also contribute
to more comprehensive and accurate analyses, uncovering new findings or patterns.

\subsubsection*{Reduce Dependency on Spotify}
Incorporating alternative sources into the data collection proccess would reduce
the reliance on Spotify's metadata and features generated using its models.
Possible options would be publicly available audio datasets, data from other
streaming platforms, or more extensive feature extraction from raw audio files.


\subsubsection*{Improve Feature Engineering Methodology}
The feature  engineering process could be greatly enhanced, with particular
focus on more advanced extraction of features from MP3 files. Deeper analysis
of importance of these features and more advanced feature selection would be
necessary. Improved feature engineering would give better insight into music's
structure and allow for exploration for more relationships and dependencies
with lyrical features, as well as contribute to more accurate analyses and
better performing models.

\subsubsection*{Implementation and Evaluation of More Advanced Modeling Techniques}
The study focused on model interpretability and for simplicity focused only on
using Catboost, since the goal was getting insight into model's predictive
processes and understanding relationships between features. More complex models
and architectures could likely strongly outperform the models trained in this
thesis and yield much more accurate results with the same set of features. Due
to computational and time constraints the searches with Optuna were also
relatively shallow, but the parameter grids could be much larger and numbers of
trials much higher with enough resources, allowing to find the best model
configurations.

\subsubsection*{Improve Clustering and Topic Modeling}
The clustering and topic modelling techniques could be largely extended and
provide much more information with enough effort. The topics extracted with LDA
could also serve as features in training of next models.
