% Informacje na stronę tytułową
\newcommand{\autor}{\fontfamily{qhv}\fontshape{n}\selectfont\LARGE\bfseries
Imię i nazwisko}
%
\newcommand{\album}{\fontfamily{qhv}\fontshape{n}\selectfont
121352}

\newcommand{\pltitle}{\fontfamily{qhv}\fontshape{n}\selectfont\Large\bfseries
Tytuł pracy po polsku}
\newcommand{\engtitle}{\fontfamily{qhv}\fontshape{n}\selectfont\Large\bfseries
Title of the thesis in English}
\newcommand{\level}{\fontfamily{qhv}\fontshape{n}\selectfont\Large
inżynierska} %jeśli magisterka to trzeba sobie przystosować pamiętając o specjalności
\newcommand{\kierunek}{\fontfamily{qhv}\fontshape{n}\selectfont\Large
Informatyka}
%
\newcommand{\promotor}{\fontfamily{qhv}\fontshape{n}\selectfont\large\bfseries
dr inż. Daniela Grzonki}
\newcommand{\spec}{\fontfamily{qhv}\fontshape{n}\selectfont\Large\bfseries
Brak}
\newcommand{\rok}{\fontfamily{qhv}\fontshape{n}\selectfont
2023}

% Strona tytułowa
\begin{titlepage}
\thispagestyle{empty}
\vspace*{1ex}
\fontfamily{qhv}\fontshape{n}\selectfont
\hspace{-3.5em}\parbox{0.15\textwidth}{\includegraphics[height=0.15\textwidth]{logo_pk/logoPK.png}}
\parbox[c]{0.7\textwidth}
{\centering
\Large
{\bfseries Politechnika Krakowska}\\
{\bfseries im. Tadeusza Kościuszki}\\

\smallskip

\Large Wydział Informatyki i Telekomunikacji}\hspace{0.3em}
 \parbox{0.2\textwidth}{\includegraphics[width=0.15\textwidth]{logo_pk/logoIT.png}}%\hspace{0.12em}
\parbox[c]{0.15\textwidth}



 \vspace{0.10\textheight}

 \center{\autor}
 \smallskip
 \center{\large numer albumu: \album}

\vspace*{2ex}

 \center{\pltitle}

 \smallskip

 \center{\engtitle}

 \bigskip

 \center{\Large\bfseries praca \level \\[0.2ex] %jeśli magisterka to trzeba sobie przystosować pamiętając o specjalności
\fontfamily{qhv}\fontshape{n}\selectfont\Large\bfseries na kierunku
 \kierunek\\[0.2ex]
 }

\vspace{0.11\textheight}

\hspace{0.57\textwidth} \parbox{0.42\textwidth}{Praca przygotowana pod
kierunkiem:
\\
%
\bfseries \promotor}

\vfill

 % wpisać rok
 \center{Kraków, \rok}
 \end{titlepage}

 \newpage

% \vspace*{17cm}
% \begin{flushright}
% \textit{Tu można sobie coś wpisać}
% \end{flushright}


