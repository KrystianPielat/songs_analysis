\chapter{Introduction}
\label{cha:introduction}

\section{Problem Statement}
\label{sec:problemstatement}
Music has been an integral part of human culture for
generations. It's been evolving alongside societies, reflecting their
creativity, emotions, values and emerging trends. In this day and age music has
become more accessible and popular than ever, especially thanks to advancements
in technology. Rapidly expanding global audience, growing number of artists and
overwhelming number of songs released every day presents us with a great
opportunity to investigate more closely its characteristics through the lens of
data-driven analyses and machine learning techniques. 

Based on the acoustic features that can be extracted from the music tracks and
textual features extracted from song lyrics we attempt to understand better
complex relationships between different music track characteristics and uncover
insights into how music is perceived by listeners, the factors that influence
song's popularity, the defining traits of various music genres and how trends
shape the evolution of music and its characteristics over the years. 

This approach offers a modern, quantitative perspective on music, enabling us
to further explore relationships between different musical and textual
characteristics.




% \begin{center}
% \begin{figure}[ht]
%   \centering
%   \includegraphics[width=6in]{img/plik.png}
%   \caption{Podpis pod rysunkiem\cite{nature_behavior}.}
%   \label{Figure:fig_beh}
% \end{figure}
% \end{center}

%---------------------------------------------------------------------------

\section{Research Objectives}
\label{sec:researchobjectives}
This thesis aims to create a robust and diverse dataset of songs with acoustic
features, lyrics and metadata through a well-defined data collection
methodology and use advanced data analysis methods and explainable artificial
intelligence(\textit{XAI}) techniques to:

\begin{itemize} 
  \item Analyze relationships between musical and textual characteristics of
    collected songs
  \item Examine how song characteristics vary across various genres and what are the defining traits
  \item Investigate the relationship between lyrical sentiment and acoustic
    dynamics (e.g., valence and tempo).
  \item Identify and understand key factors contributing to song's popularity
  \item Explore music characteristics evolution over the years and shifts in listener's preferences
  \item Build predictive models to predict various song attributes and use XAI
    to interpret the models and extract meaningful insights and feature
    relationships
\end{itemize}

%---------------------------------------------------------------------------

\section{Research Questions}
\label{sec:researchquestions}
Using collected data and established methodologies following questions will be
considered:

\begin{enumerate}
  \item \textbf{Acoustic and Textual Relationships} - how does acoustic
    properties like energy and tempo relate to lyrical features like sentiment?
  \item \textbf{Genre-specific traits} - What makes different genres unique in terms of their sound and lyrics?
    properties like energy and tempo relate to lyrical features like sentiment?
  \item \textbf{Popularity Determinants} - What are the features that have
    strongest influence on song's popularity?
\end{enumerate}

%---------------------------------------------------------------------------


\section{Thesis Structure}
\label{sec:thesisstructure}

The structure is organized to address specified research objectives. It begins
with introduction to the methodologies used for data collection, preprocessing,
feature extraction, followed by exploration of the analysis and XAI techniques.
Subsequent chapters focus on the verification of hypotheses with various
analytical approaches. Finally the thesis concludes with a discussion of
results and potential directions for future work.

%---------------------------------------------------------------------------
