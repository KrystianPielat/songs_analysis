\chapter{Introduction}
\label{cha:introduction}

\section{Problem Statement}
\label{sec:problemstatement}
Music has been an integral part of human culture for
generations. It's been evolving alongside societies, reflecting their
creativity, emotions, values and emerging trends. In this day and age music has
become more accessible and popular than ever, especially thanks to advancements
in technology. Rapidly expanding global audience, growing number of artists and
overwhelming number of songs released every day presents us with a great
opportunity to investigate more closely its characteristics through the lens of
data-driven analyses and machine learning techniques. 

Based on the acoustic features that can be extracted from the music tracks, and
textual features extracted from song lyrics and information available on
Spotify, this paper attempts to understand better complex relationships between
different music track characteristics and uncover insights into how music is
perceived by listeners, the factors that influence song's popularity, the
defining traits of various music genres and how trends shape the evolution of
music and its characteristics over the years. 

This approach offers a modern, quantitative perspective on music, enabling us
to further explore relationships between different musical and textual
characteristics.




% \begin{center}
% \begin{figure}[ht]
%   \centering
%   \includegraphics[width=6in]{img/plik.png}
%   \caption{Podpis pod rysunkiem\cite{nature_behavior}.}
%   \label{Figure:fig_beh}
% \end{figure}
% \end{center}

%---------------------------------------------------------------------------

\section{Research Objectives}
\label{sec:researchobjectives}
This thesis aims to create a robust and diverse dataset of songs with acoustic
features, lyrics and metadata through a well-defined data collection
methodology and use advanced data analysis methods and explainable artificial
intelligence(\textit{XAI}) techniques to:

\begin{itemize} 
  \item Analyze relationships between musical and textual characteristics of
    collected songs
  \item Examine how song characteristics vary across various genres and what
    are their defining traits
  \item Investigate the relationship between lyrical sentiment and acoustic
    dynamics (e.g., valence and tempo).
  \item Identify and understand key factors contributing to song's popularity
  \item Explore music characteristics evolution over the years and shifts in listener's preferences
  \item Build predictive models to predict various song attributes and use XAI
    to interpret the models and extract meaningful insights and feature
    relationships
\end{itemize}

Additionally, this thesis aims to create a practical and reusable framework for
explainable AI (\textit{XAI}) evaluation, as well as a clear and systematic
method for collecting data. These tools are designed to make it easier for
future researchers to build on this work, apply the methods to new projects,
and explore other areas of music analysis.

%---------------------------------------------------------------------------

\section{Research Questions}
\label{sec:researchquestions}
This thesis aims to leverage a diverse dataset of songs collected through a
systematic metehodology, consisting of metadata, Spotify's audio features,
lyrics and mp3 tracks. With the help of advanced data analysis techniques and
explainable artificial intelligence it attempts to achieve following
objectives:
\begin{enumerate}
  \item Explore the relationships between lyrical, acoustic, and metadata
    features of songs to uncover meaningful patterns and dependencies.
  \item Analyze defining characteristics of songs across different genres and
    investigate how these features contribute to genre identity.
  \item Develop predictive models to estimate various song attributes (e.g.
    popularity, explicitness, sentiment) and use XAI to interpret these models,
    extracting meaningful insights about feature importance and relationships.
  \item Identify commonly recurring themes in lyrics and explore their
    significance and distinctive traits. 
  \item Uncover temporal trends and shifts in music characteristics over time.
\end{enumerate}

%---------------------------------------------------------------------------


\section{Thesis Structure}
\label{sec:thesisstructure}

This thesis is structured into several chapters to address the research
objectives. It begins with an introduction, followed by a discussion of related
work, highlighting key studies and their connection to this research. The data
chapter explains the sampling method, data collection framework, and sources
used.

Next, the methodology chapter details the feature engineering process,
describing all extracted features, and introduces the explainable AI
(\textit{XAI}) techniques and other methods applied in the study. This is
followed by an extensive data analysis chapter, exploring relationships between
features and identifying distinctive traits of each genre.

The experiments chapter presents the conducted experiments and discussion of
the results. The thesis concludes with a summary of contributions, a description
of the limitations, key findings, and recommendations for future work. 

