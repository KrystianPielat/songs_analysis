\chapter{Introduction}
\label{cha:introduction}

\section{Problem Statement}
\label{sec:problemstatement}
Music has been a prominent part of human culture for generations. It's been
evolving alongside societies, reflecting their creativity, emotions, values and
emerging trends. In this day and age music has become more accessible and
popular than ever, especially thanks to advancements in technology. Rapidly
expanding global audience, growing number of artists and overwhelming number of
songs released every day presents us with a great opportunity to investigate
more closely its characteristics through the lens of data-driven analyses and
machine learning techniques. 

Based on the acoustic features that can be extracted from the music tracks, and
textual features extracted from song lyrics and information available on
Spotify, this paper attempts to understand better complex relationships between
different music track characteristics and uncover insights into how music is
perceived by listeners, the factors that influence song's popularity, the
defining traits of various music genres and how trends shape the evolution of
music and its characteristics over the years. 

This approach offers a modern, quantitative perspective on music, enabling us
to further explore relationships between different musical and textual
characteristics.




% \begin{center}
% \begin{figure}[ht]
%   \centering
%   \includegraphics[width=6in]{img/plik.png}
%   \caption{Podpis pod rysunkiem\cite{nature_behavior}.}
%   \label{Figure:fig_beh}
% \end{figure}
% \end{center}

%---------------------------------------------------------------------------

\section{Research Questions \& Objectives}
\label{sec:researchquestions}
% This thesis aims to leverage a diverse dataset of songs collected through a
% systematic metehodology, consisting of metadata, Spotify's audio features,
% lyrics and mp3 tracks. 

The aim of this thesis is to build a diverse dataset of songs with lyrical,
acoustic and metadata features through a well-defined data collection
methodology and use various advanced data analysis methods and explainable
artificial intelligence(\textit{XAI}) to:

\begin{enumerate}
  \item Explore the relationships between lyrical, acoustic, and metadata
    features of songs to uncover meaningful patterns and dependencies.
  \item Analyze defining characteristics of songs across different genres and
    investigate how these features contribute to genre identity and what are
    defining traits for each genre included in the dataset.
  \item Develop predictive models to estimate various song attributes (e.g.
    popularity, explicitness, sentiment) and use XAI methodologies to interpret
    model's predictions and extract meaningful insights and relationships.
  \item Identify commonly recurring themes in lyrics and explore their
    significance and distinctive traits using topic modelling. 
  \item Uncover temporal trends and shifts in music characteristics over time.
\end{enumerate}

Additionally, this thesis aims to create a practical and reusable explainable
AI(\textit{XAI}) evaluation framework and a clear and systematic method for
collecting data. These tools are designed to make it easier to build on this
work, apply the methods to new projects and research.


%---------------------------------------------------------------------------


\section{Thesis Structure}
\label{sec:thesisstructure}

This thesis begins with an introduction and explaination of research objectives.
It's followed by a discussion of related studies, summarizing each of them and
explaining how  they relate to this work. 

The data chapter explains the sampling method that was used to select the songs
from Spotify, the data collection framework, the sources for the data and the
data itself.

Next, the methodology chapter outlines the feature engineering process,
describing all extracted features, and introduces all the techniques
and methods used in this study. 

The chapter that comes after that describes the extensive exploratory data
analysis conducted on the collected data with the aim of exploring relationships
between features and identifying distinctive traits of each genre.

The experiments chapter presents all the experiments conducted in this study
and discusses their results. It also compares them to different benchmarks and
attempts to interpret the results.

The thesis concludes with a summary of contributions, a description
of the limitations, key findings, and recommendations for future work. 
