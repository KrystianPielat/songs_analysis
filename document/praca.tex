\documentclass[oneside, 12pt]{book}
\usepackage[polish]{babel}
\usepackage{geometry}
\usepackage{lipsum}
\usepackage{nameref}
\usepackage{polski}
\usepackage[utf8]{inputenc}
\usepackage{setspace}
\usepackage[export]{adjustbox}
\usepackage{biblatex}
\usepackage{titlesec}
\usepackage{csquotes}
\usepackage{float}
\usepackage[pdfsubject={Pielat K.: Analysis of the Influence of Acoustic
Parameters and Lyrics on the Characteristics of Music Tracks Using AI Methods},
pdfauthor={Pielat K.},
pdftitle={Pielat K.: Analysis of the Influence of Acoustic Parameters and
Lyrics on the Characteristics of Music Tracks Using AI Methods},
pdfkeywords={musical analysis; nlp; acoustic analysis}]{hyperref}


\setcounter{tocdepth}{3}
\setcounter{secnumdepth}{3}

\addbibresource{bibliography.bib}

\titleformat{\chapter}[block]
  {\normalfont\huge\bfseries}{\thechapter.}{1em}{\Huge}

\titlespacing*{\chapter}{0pt}{50pt}{40pt}

\geometry{
    top=1in,
    bottom=1in,
    outer=1in,
    inner=1in,
}

\pagestyle{plain} %Numeracja na dole

\linespread{1.5} % Interlinia 1.5

\setlength{\parindent}{1.25cm} % Wcięcie akapitu 1.25cm

\begin{document}
\thispagestyle{empty}

% Informacje na stronę tytułową
\newcommand{\autor}{\fontfamily{qhv}\fontshape{n}\selectfont\LARGE\bfseries
Krystian Pielat}
%
\newcommand{\album}{\fontfamily{qhv}\fontshape{n}\selectfont
143007}

\newcommand{\pltitle}{\fontfamily{qhv}\fontshape{n}\selectfont\Large\bfseries
Analiza wpływu parametrów akustycznych i tekstów na charakterystykę utworów muzycznych z wykorzystaniem metod sztucznej inteligencji}
\newcommand{\engtitle}{\fontfamily{qhv}\fontshape{n}\selectfont\Large\bfseries
 Analysis of the Influence of Acoustic Parameters and Lyrics on the Characteristics of Music Tracks Using AI Methods}
\newcommand{\level}{\fontfamily{qhv}\fontshape{n}\selectfont\Large
inżynierska} %jeśli magisterka to trzeba sobie przystosować pamiętając o specjalności
\newcommand{\kierunek}{\fontfamily{qhv}\fontshape{n}\selectfont\Large
Informatyka}
%
\newcommand{\promotor}{\fontfamily{qhv}\fontshape{n}\selectfont\large\bfseries
dr inż. Daniela Grzonki}
\newcommand{\spec}{\fontfamily{qhv}\fontshape{n}\selectfont\Large\bfseries
Brak}
\newcommand{\rok}{\fontfamily{qhv}\fontshape{n}\selectfont
2024}

% Strona tytułowa
\begin{titlepage}
\thispagestyle{empty}
\vspace*{1ex}
\fontfamily{qhv}\fontshape{n}\selectfont
\hspace{-3.5em}\parbox{0.15\textwidth}{\includegraphics[height=0.15\textwidth]{logo_pk/logoPK.png}}
\parbox[c]{0.7\textwidth}
{\centering
\Large
{\bfseries Politechnika Krakowska}\\
{\bfseries im. Tadeusza Kościuszki}\\

\smallskip

\Large Wydział Informatyki i Telekomunikacji}\hspace{0.3em}
 \parbox{0.2\textwidth}{\includegraphics[width=0.15\textwidth]{logo_pk/logoIT.png}}%\hspace{0.12em}
\parbox[c]{0.15\textwidth}



 \vspace{0.10\textheight}

 \center{\autor}
 \smallskip
 \center{\large numer albumu: \album}

\vspace*{2ex}

 \center{\pltitle}

 \smallskip

 \center{\engtitle}

 \bigskip

 \center{\Large\bfseries Praca \level \\[0.2ex] %jeśli magisterka to trzeba sobie przystosować pamiętając o specjalności
\fontfamily{qhv}\fontshape{n}\selectfont\Large\bfseries na kierunku
 \kierunek\\[0.2ex]
 }

\vspace{0.11\textheight}

\hspace{0.57\textwidth} \parbox{0.42\textwidth}{Praca przygotowana pod
kierunkiem:
\\
%
\bfseries \promotor}

\vfill

 % wpisać rok
 \center{Kraków, \rok}
 \end{titlepage}

 \newpage

% \vspace*{17cm}
% \begin{flushright}
% \textit{Tu można sobie coś wpisać}
% \end{flushright}




\tableofcontents
\clearpage

\chapter{Introduction}
\label{cha:introduction}

\section{Problem Statement}
\label{sec:problemstatement}
Music has been a prominent part of human culture for generations. It's been
evolving alongside societies, reflecting their creativity, emotions, values and
emerging trends. In this day and age music has become more accessible and
popular than ever, especially thanks to advancements in technology. Rapidly
expanding global audience, growing number of artists and overwhelming number of
songs released every day presents us with a great opportunity to investigate
more closely its characteristics through the lens of data-driven analyses and
machine learning techniques. 

Based on the acoustic features that can be extracted from the music tracks, and
textual features extracted from song lyrics and information available on
Spotify, this paper attempts to understand better complex relationships between
different music track characteristics and uncover insights into how music is
perceived by listeners, the factors that influence song's popularity, the
defining traits of various music genres and how trends shape the evolution of
music and its characteristics over the years. 

This approach offers a modern, quantitative perspective on music, enabling us
to further explore relationships between different musical and textual
characteristics.




% \begin{center}
% \begin{figure}[ht]
%   \centering
%   \includegraphics[width=6in]{img/plik.png}
%   \caption{Podpis pod rysunkiem\cite{nature_behavior}.}
%   \label{Figure:fig_beh}
% \end{figure}
% \end{center}

%---------------------------------------------------------------------------

\section{Research Questions \& Objectives}
\label{sec:researchquestions}
% This thesis aims to leverage a diverse dataset of songs collected through a
% systematic metehodology, consisting of metadata, Spotify's audio features,
% lyrics and mp3 tracks. 

The aim of this thesis is to build a diverse dataset of songs with lyrical,
acoustic and metadata features through a well-defined data collection
methodology and use various advanced data analysis methods and explainable
artificial intelligence(\textit{XAI}) to:

\begin{enumerate}
  \item Explore the relationships between lyrical, acoustic, and metadata
    features of songs to uncover meaningful patterns and dependencies.
  \item Analyze defining characteristics of songs across different genres and
    investigate how these features contribute to genre identity and what are
    defining traits for each genre included in the dataset.
  \item Develop predictive models to estimate various song attributes (e.g.
    popularity, explicitness, sentiment) and use XAI methodologies to interpret
    model's predictions and extract meaningful insights and relationships.
  \item Identify commonly recurring themes in lyrics and explore their
    significance and distinctive traits using topic modelling. 
  \item Uncover temporal trends and shifts in music characteristics over time.
\end{enumerate}

Additionally, this thesis aims to create a practical and reusable explainable
AI(\textit{XAI}) evaluation framework and a clear and systematic method for
collecting data. These tools are designed to make it easier to build on this
work, apply the methods to new projects and research.


%---------------------------------------------------------------------------


\section{Thesis Structure}
\label{sec:thesisstructure}

This thesis begins with an introduction and explaination of research objectives.
It's followed by a discussion of related studies, summarizing each of them and
explaining how  they relate to this work. 

The data chapter explains the sampling method that was used to select the songs
from Spotify, the data collection framework, the sources for the data and the
data itself.

Next, the methodology chapter outlines the feature engineering process,
describing all extracted features, and introduces all the techniques
and methods used in this study. 

The chapter that comes after that describes the extensive exploratory data
analysis conducted on the collected data with the aim of exploring relationships
between features and identifying distinctive traits of each genre.

The experiments chapter presents all the experiments conducted in this study
and discusses their results. It also compares them to different benchmarks and
attempts to interpret the results.

The thesis concludes with a summary of contributions, a description
of the limitations, key findings, and recommendations for future work. 

\chapter{Related Work}
\label{cha:literaturereview}

%---------------------------------------------------------------------------

\section{Acoustic Features in Music Analysis}
\label{sec:acousticfeaturesinmusicanalysis}

\subsubsection*{``Music Genre Classification Using MFCC, k-NN, and SVM
Classifier''}


This study explores music genre classification using MFCCs and Chroma features
on the GTZAN dataset, that consists of 900 tracks across 9 genres. The
best-performing model was an SVM with a polynomial kernel, achieving accuracy
of 78\%. Some genres were identified to have overlapping characteristics which
posed problems for the classification model. \cite{music_genre_classification_mfcc}

The paper highlights the effectiveness of MFCCs and Chroma features for
audio-based classification, aligning with this thesis's use of acoustic
features. However, unlike this study, the thesis extends the analysis by
incorporating lyrical features, metadata, and audio features provided by
Spotify, addressing limitations in feature diversity.

\subsubsection*{``Classifying Music Audio with Timbral and Chroma Features''}

The paper ``Classifying Music Audio with Timbral and Chroma Features'' examines
the use of timbral (e.g., MFCCs) and harmonic (e.g., chroma) features in music
classification tasks. The study highlights that MFCCs, which represent timbral
qualities, are effective for tasks like \textbf{artist identification, achieving 56\%
accuracy in a 20-way classification task}. When combined with beat-synchronous
chroma features, which capture harmonic and melodic information, the model's
accuracy improved to 59\%. This demonstrates the importance of leveraging
complementary audio features for more robust classification. These findings
emphasize the potential of combining diverse acoustic features in predictive
models, aligning closely with the objectives of this thesis. \cite{classifying_music_audio}



%---------------------------------------------------------------------------

\section{Lyrical Features in Music Analysis}
\label{sec:lyricalfeaturesinmusicanalysis}

\subsubsection*{``Using Machine Learning Analysis to Interpret the Relationship
Between Music Emotion and Lyric Features''}


This study investigates the relationship between lyrical features and perceived
music emotions using 2,372 Chinese pop songs. Lyric features were extracted
with LIWC (Linguistic Inquiry and Word Count), and audio features
such as MFCCs and chroma were derived using Librosa. The analysis highlighted
that lyrical features like the frequency of positive and negative emotion words
contributed significantly to predicting the perceived valence of music, whereas
audio features were dominant in predicting perceived arousal. \cite{valence_and_lyrics}

The study utilized Random Forest regression models, demonstrating that
combining lyric and audio features improved \textbf{predictions of valence,
with an $R^2$ of 0.481}, but lyrics had little impact on arousal models. These
findings align with this thesis's use of both lyrical and acoustic features for
predictive tasks but differ in the application of explainable AI (XAI)
techniques like SHAP for feature interpretation. Moreover, this thesis expands
this study by incorporating Spotify audio features and metadata for broader
analytical capabilities and more diverse research objectives.


\subsubsection*{``Sentiment Analysis and Lyrics Theme Recognition Using NLP Techniques''}


This paper investigates the relationship between sentiment and themes in music
lyrics using Natural Language Processing (NLP) techniques. It applies sentiment
analysis and thematic recognition across a diverse dataset, identifying
emotional nuances and recurring topics in the lyrics. The analysis identifies
correlations between sentiment categories (positive, negative, neutral) and
thematic clusters (e.g., love, social justice, personal reflection). The
authors use techniques like Latent Dirichlet Allocation (LDA) for topic
modeling and Support Vector Machines (SVM) for sentiment classification. \cite{du_2024}

In the context of this thesis, this study aligns closely with the focus on
lyrical analysis, particularly in employing NLP-driven sentiment and thematic
classification. However, while this work emphasizes standalone lyric-based
analysis, this thesis extends the methodology by integrating Spotify’s audio
features, acoustic analysis, and metadata. 

%---------------------------------------------------------------------------

\section{Machine Learning in Music Analysis}
\label{sec:machinelearningfeaturesinmusicanalysis}

\subsubsection*{``Beyond Beats: A Recipe to Song Popularity? A Machine Learning Approach''}

The paper ``Beyond Beats'' investigates the predictive power of various machine
learning models for song popularity on a dataset of 30,000 songs spanning six
genres. It focuses on metadata and audio features fetched from
Spotify (\textit{danceability}, \textit{acousticness} etc.), as well as the
genre. The best performing model was \textbf{Random Forest and achieved 16.31 MAE.}
Predictions across all methods remained relatively modest, reflecting the
complex and multi-dimensional  nature of song popularity. Those results can be
greatly improved using additional features. The authors noted that predictive
accuracy was constrained by the absence of post-release factors such as
marketing, social media reception, and artist reputation. \cite{beyond_beats}


\subsubsection*{``Predicting Song Popularity in the Digital Age Through
Spotify’s Data''}  
Similarily to the previous one, the paper ``Predicting Song Popularity in the
Digital Age Through Spotify’s Data'' explores the relationship between
Spotify's audio features  and song popularity, using a dataset spanning from
1986 to 2022. The study employed linear regression to predict popularity and
\textbf{achieved an adjusted $R^2$ of 0.38}, highlighting the moderate
predictive power of these features. The analysis revealed that attributes such
as \textit{danceability} and \textit{duration} positively correlate with
popularity, while \textit{speechiness} tends to have a negative
impact. \cite{predicting_song_popularity_2024}

\chapter{Data}
\label{cha:data}
%---------------------------------------------------------------------------

\section{Dataset Description}
\label{sec:datasetdescription}

The dataset consists of around 6000 songs. For each song metadata, audio
recording, lyrics and spotify audio features were fetched.

\section{Data Collection Methods}
\label{sec:datacollectionmethods}

\subsection{Sources}
The data was collected from various sources. Metadata and audio features
were fetched from Spotify API. The lyrics were scraped from MusixMatch,
LetrasMus, MakeItPersonal or Lyrics Fandom, or fetched via the Genius API,
depending on  the availability. Audio files were downloaded from YouTube and
saved as mp3 files.

\subsection{Methodology}
The data collection process was automated for efficiency. The starting point
was a list of Spotify playlist URIs. Each playlist was processed sequentially. 
The script fetched metadata and audio features for each song in the
playlist.

Lyrics were then searched based on \textit{artist name} and \textit{title} of
each song, using multiple lyrics providers. The system continued to query
different sources until it found lyrics in at least one of them. If it failed
to retrieve lyrics for the song, it was discarded and wouldn't make it to the
final dataset.

Finally, the script  searched YouTube for each song and downloaded the first
relevant result, saving it to an mp3 file. All data was saved into a CSV
file named after the playlist and stored alongside the recordings.

The entire process was parallelized to significantly enhance the speed of data
acquisition. It was designed in a robust manner, with careful error handling
and ability to stop the process at any  time and pick up where it left off.

Additionally, to optimize performance, the entire process was parallelized,
significantly increasing the speed of data acquisition. It was designed with
robust error handling, ensuring data correctness and completeness. Moreover, it
featured the ability to pause and seamlessly resume the process, continuing
exactly where it left off without data loss.


%---------------------------------------------------------------------------


\subsection{Metadata}
\label{sec:metadata}
Song's information downloaded from Spotify API. The information it includes is:

\begin{itemize}
  \item \textbf{Popularity} - relative measure with values ranging from 0 to
    100 describing how popular the song is, estimated mostly based on total
    number of plays and how recent  those plays are
  \item \textbf{Explicitness} - whether or not the song contains explicit lyrics
  \item \textbf{Genre} - main genre of the artist(Spotify does not provide
    information about genre of each musical track)
  \item \textbf{Album Release Year} - the release year of the album that the
    song originates from
\end{itemize}

\textit{popularity}, \textit{explicitness},
\textit{genre}, and \textit{release year}. Since Spotify does not provide
information about the genre of specific songs, the genre extracted is the main
genre of the primary artist in the song.


\subsection{Spotify Audio Features}
\label{sec:spotifyaudiofeatures}
Those features were fetched from the Spotify API.They describe different
acoustic properties songs:
\begin{itemize}
  \item \textbf{Speechiness} - relative measure of spoken words in a track
  \item \textbf{Acousticness} - a confidence measure of whether the song is
    acoustic
  \item \textbf{Danceability} - a measure of how suitable the song is for
    dancing. Its based on parameters like tempo, rhythm stability, beat
    strength etc.
  \item \textbf{Energy} - a measure of perceived intensity of songs. Energetic
    tracks are usually louder, faster, feel more intense.
  \item \textbf{Loudness} - overall loudness of the  track in dB averaged
    across the entire track
  \item \textbf{Valence} - relative measure describing musical positiveness of
    a track
  \item \textbf{Instrumentalness} - measure of how likelihood of the track not
    containing vocals. In this paper since lyrics are mandatory it's used to
    discard instrumental tracks.
  \item \textbf{Liveness} - probability of the song being recorded during a
    live performance.
  \item \textbf{Key} - the key of the song, e.g. C\#.
  \item \textbf{Mode} - indicates the modality of a track(major / minor)
  \item \textbf{Tempo} - estimated tempo of a track in beats per minute(BPM)
  \item \textbf{Time Signature} - specifies how many beats there are in each
    bar
  \item \textbf{Duration} - the duration of the track in milliseconds
\end{itemize}


\subsection{Lyrics Features}
 The lyrics serve as a textual representation of the song's thematic,
 emotional, and linguistic elements. Since they came from various data sources,
 they had  to undergo cleaning procedure in order to remove faulty information
 and prepare them to be processed by the textual feature extraction class. The
 process consisted of:
 \begin{itemize}
  \item \textbf{Standardization} - lyrics were converted to lowercase
  \item \textbf{Noise Removal} - unnecessary characters, numbers and
    punctuation, as well as additional comments used by lyrics providers(e.g.
    'chorus') were removed
  \item \textbf{Stopwords Filtration} - exclusion of frequently occuring words
    that carry little information, like 'the' in English
  \item \textbf{Stemming} - words were reduced to their root forms to enhance
    uniformity and reduce corpus size
 \end{itemize}

 This process laid foundation for further extraction of textual features used
 for exploratory data analysis, statistical inference and training ML models.
 That process will be explained in later chapter.


%---------------------------------------------------------------------------

\section{Tools and Libraries Used}
\label{sec:toolsandlibrariesused}
Python libraries used to facilitate the data acquisition were:
\begin{itemize}
  \item \textit{Spotipy} - a lightweight python library for Spotify API
  \item \textit{youtube-dl} - a library used to find and download youtube videos
  \item \textit{BeautifulSoup} - a library used for extracting information from
    HTML, commonly used for web scraping
\end{itemize}




\chapter{Methodologies}
\label{cha:methodologies}


This chapter explains the methodologies used to address the research objectives
in this thesis. It describes the feature extraction and  engineering methods,
the use of explainable AI methods, model training and optimization approaches,
clustering and dimensionality reduction techniques and statistical hypothesis
testing methods.
%---------------------------------------------------------------------------

\section{Feature Engineering}
\label{sec:featureengineering}

Feature engineering is a critical step in the research process, as it involves
transforming raw data acquired using the data collection script into meaningful
representations that can be analyzed or used for predictive modeling. This
section describes the methodologies used to extract acoustic features from the
mp3 files and lyrical features from the lyrics. These features are designed to
capture key characteristics of the songs, enabling deeper insights into their
patterns and relationships.

\subsection{Acoustic Features}
\label{sec:acousticfeatures}

 These features provide a quantitative representation of the audio properties
 of each song and were extracted directly from the audio files in MP3 format.
 They describe various aspects of audio signal and provide insights into the
 rhythm, timbre, harmony and other acoustic properties. The extraction was done
 using \textit{Librosa} and was automated and parallelized to make it suitable
 for processing large amounts of data. 

\subsubsection*{MFCC - Mel Frequency Cepstral Coefficients}
MFCCs represent the short-term power spectrum of a song on a mel-scale and are
widely used for timbre analysis. These coefficients capture the tonal quality
of the audio and help differentiate between different instruments and vocal
characteristics.


\subsubsection*{Chroma}
Chroma vectors represent the intensity of each pitch class (e.g., C, C\#, D,
etc.) in the audio. These features provide a harmonic representation of the
song and are useful for analyzing chord progressions and harmonic structures.

\subsubsection*{Spectral Contrast}
Spectral contrast measures the difference in amplitude between peaks and
valleys in the spectrum. It provides insights into the harmonic and timbral
content of a song, particularly useful for distinguishing between smooth and
complex textures.

\subsubsection*{Other Features}
Two additional features were extracted:
\begin{itemize}
  \item \textbf{Tempo} - refers to the speed of the song, measured in \textit{Beats Per Minute(BPM)}
  \item \textbf{Zero Crossing Rate(ZCR)} - measures the rate at which the audio
    signal changes sign. It's commonly used as a measure of noisiness or
    percussive nature of signal
\end{itemize}

%---------------------------------------------------------------------------

\subsection{Lyrical Features}
\label{sec:lyricalfeatures}

Lyrical features  were extracted from the lyrics fetched during the data
collection process. They aim to provide a linguistic and semantic
representation of the track, capturing their complexity, sentiment and
stylistic  attributes. The cleaning and extraction  process utilized various
NLP libraries like \textit{NLTK}, \textit{spaCy} and \textit{TextBlob},
alongside with custom ad-hoc algorithms. Similarily to acoustic features the
implementation allowed for simple and intuitive usage under clear and
comprehensible interface, with parallelization of the computation process for
increased performance. The features extracted can be grouped as follows:

\subsubsection*{Basic Linguistic Metrics}
\begin{itemize}
  \item \textbf{Unique Word Count} - measures number of unique words in the lyrics, indicating diversity
  \item \textbf{Type-Token Ratio} - a measure of lexical richness: ratio of unique words to total words
  \item \textbf{Word Count} - total  number of words, baseline for text size and compexity
  \item \textbf{Noun and Verb Ratios} - proporrtios of nous and vers relative to the total word count
\end{itemize}


\subsubsection*{Sentiment and Emotional Tone}
\begin{itemize}
  \item \textbf{Sentiment Polarity} - a measure of overall sentiment(positive
    vs. negative) of the text
  \item \textbf{Sentiment Subjectivity} - represents the degree of subjectivity
    in the lyrics, attempting to make a distinction between factual and
    opinionated content
  \item \textbf{VADER Compound} - a sentiment score derived from the VADER tool
  \item \textbf{Sentiment Variability} - standard deviation of sentiment on
    subsets of lyrics, a metric  aiming to capture fluctuations of sentiment
    throughout the song, highlighting emotional complexity
  \item \textbf{} -
\end{itemize}


\subsubsection*{Stylistic Features}
\begin{itemize}
  \item \textbf{Repetition Count} - the frequency of repeated words
  \item \textbf{Rhyme Density} - a measure of how often rhymes occur in the
    text
  % \item \textbf{Linguistic Uniqueness} - a measure of 
\end{itemize}


\subsubsection*{Semantic and Complexity Features}
\begin{itemize}
  \item \textbf{Semantic Depth} - represents the richness and variety of
    meaning conveyed by the lyrics
  \item \textbf{Syntactic Complexity} - captures the sophistication of
    sentence structures
  \item \textbf{Lexical Richness} - quantifies the variety and richness of the
    vocabulary
\end{itemize}



\subsubsection*{Readability and Accessibility}
\begin{itemize}
  \item \textbf{Flesch Reading Ease} - indicates how easy the lyrics are to
    read
  \item \textbf{Gunning Fog} - a metric that estimates the  years of education
    required to understand the text
  \item \textbf{Dale Chall} - a metric that accounts for familiar and
    unfamiliar words in the text
\end{itemize}


\subsubsection*{Contextual Information}
  In process of feature extraction the \textbf{language} of lyrics was also
  identified using \textit{langdetect} library that uses a classification model
  to make predictions based on n-grams extracted from the text. The identified
  language was also used in the cleaning process, to identify which stemmer and
  stopwords language  to use.


\subsubsection*{TF-IDF (Term Frequency - Inverse Document Frequency)}

TF-IDF\cite{tfidf} is a measure that can quantify the relevance of tokens in a document
amongst a collection of documents. In the context of this study, it determines
how important a word in a song's lyrics is  compared to all other song's lyrics
in the dataset. It's used to highlight words that are unique or meaningful
while giving less importance to very common words like 'the' or 'and'.

It can be broken down into two parts:


\begin{itemize}
  \item \textbf{TF - Term Frequency} - Measures the frequency of a term within
    a document:
  \[
  TF(w, d) = \frac{f_{w, d}}{N_d}
  \]
  where:
  \begin{itemize}
    \item \( f_{w, d} \): The number of times the word \( w \) appears in the document \( d \).
    \item \( N_d \): The total number of words in the document \( d \).
  \end{itemize}

  \item \textbf{IDF - Inverse Document Frequency} - Measures the rarity of a
    term across a collection of documents:
  \[
  IDF(w) = \log{\frac{N}{1 + n_w}}
  \]
  where:
  \begin{itemize}
    \item \( N \): The total number of documents in the corpus.
    \item \( n_w \): The number of documents containing the word \( w \).
  \end{itemize}
\end{itemize}

%---------------------------------------------------------------------------

\section{Explainable AI Methods}
\label{sec:explainableaimethods}


Explainable AI (XAI) techniques provide insights into the decision-making
processes of ML models, making it possible to understand the complex
relationships captured within the training data. By bridging the gap between
the pattern-recognition capabilities of these models and their practical
applications, XAI enables transparency and improves the interpretability of
results.

In this study this methodology was applied in various experiments to understand
how specific variables influence others, with aim of uncovering rrelationships
in the data and validating hypotheses. A deeper  understanding of factors
driving model predictions ensured that the results were both reliable and
meaningfully adressed the research objectives. The  techniques employed in this paper are: 

\subsubsection*{SHAP (SHapley Additive exPlanations)}
SHAP values offer a detailed  breakdown of how individual features contribute
to  each prediction. It uses game theory principles to compute the importance
of each feature for a given output, providing both global insights on general
feature importance and  local explainations, showing how diffrenet features
contributed to specific predictions. In this study SHAP values were used in
order to explain the predictive process of trained Catboost models, allowing
for identification of key features for specific prediction task and
visualization of relationships.

\begin{center}
\begin{figure}[ht]
  \centering
  \includegraphics[width=4in]{img/shap_intro.png}
  \caption{SHAP}
  \label{Figure:fig_beh}
\end{figure}
\end{center}

\begin{center}
\begin{figure}[ht]
  \centering
  \includegraphics[width=4.5in]{img/shap_beeswarm.png}
  \caption{Example SHAP beeswarm plot showing impact of some lyrical features
  on the classifier of \textit{explicitness}}
  \label{Figure:fig_beh}
\end{figure}
\end{center}

\begin{center}
\begin{figure}[ht]
  \centering
  \includegraphics[width=3in]{img/shap_feature_importance.png}
  \caption{Example SHAP feature importance plot showing impact of some lyrical features
  on the classifier of \textit{explicitness}}
  \label{Figure:fig_beh}
\end{figure}
\end{center}
%---------------------------------------------------------------------------

\subsubsection*{Machine Learning Models}
In this study, CatBoost, a widely recognized gradient boosting algorithm known
for its high predictive accuracy and efficiency, was employed to build
classification and regression models. These models utilized a combination of
acoustic, lyrical and metadata features in order to predict the variable of
interest, leveraging Catboost's strengths in handling diverse and complex
datasets. The result models were then subjected to SHAP analysis, in order to
understand their decision process and understand the interactions between
features and the target variable. Catboost was chosen for its flexibility, ease
of use, robust performance, compatibility with SHAP and built-in support for
categorical features,  which eliminated the need for extensive preprocessing.

In order to further optimize the performance of these models, the
hyperparameter tuning library \textit{Optuna} was used. Its efficient
optimization framework allowed for systematic exploration of different sets of
hyperparameter configurations, ensuring the model achieved optimal performance while avoiding
overfitting. 

To address the challenges commonly encountered when training ML models on
complex datasets, following techniques were employed:
\begin{itemize}
  \item \textbf{Cross-validation} - cross-validation was used to reduce the
    risk of overfitting and provide reliable performance metrics. By dividing
    the data into multiple folds, the model was iteratively trained and
    validated on different subsets, therefore ensuring robust evaluation across
    the dataset and improved model's reliability, at the cost of increased
    computational time.
  \item \textbf{Class Weights} - to handle class imbalance in target variable
    labels, CatBoost offers a built-in capability to assign different penalties
    for misclassifications of specific classes. This adjustment improves
    model's ability to make accurate predictions across all classes, instead of
    favouring the majority class.
  \item \textbf{Out of sample evaluation} - model's performance was assessed on
    a separate test dataset that was excluded from the training. This step
    provided a reliable measure of model's ability to generalize on unseen data
    and ensured evaluation metrics  reflected its real predictive performance.
\end{itemize}

\section{Dimensionality Reduction - Principal Component Analysis (PCA)}
\label{sec:dimensionalityreduction}

Principal Component Analysis (PCA) reduces the number of features in large
datasets by transforming them into principal components that retain most of the
original information. It achieves this by converting potentially correlated
variables into a smaller set of less correlated variables, called principal
components, in a way that preserves as much of the original variance as
possible\cite{pca}. PCA is often employed to reduce dataset dimensionality and improve
generalization by reducing noise and redundancy in the data.

In this study, PCA was applied to the TF-IDF vectors derived from song lyrics.
TF-IDF vectors are typically high-dimensional, with thousands of features
representing individual terms across the corpus. Such high-dimensional data can
pose challenges, including increased computational complexity and a higher risk
of overfitting in machine learning models.

The use of PCA on these vectors reduced their dimensionality while preserving
as many significant patterns from the original vectors as possible. This
process improved computational efficiency and the interpretability of the data,
which is particularly important in the context of Explainable AI (XAI)
methodologies.

\begin{center}
\begin{figure}[ht]
  \centering
  \includegraphics[width=4in]{img/pca.png}
  \caption{A scatterplot showing the relationship between PC1 and PC2 when PCA
  is applied to a dataset. PC1 and PC2 axis are perpendicular to each other.\cite{pca}}
  \label{Figure:fig_beh}
\end{figure}
\end{center}


\section{Topic Modelling - Latent Dirichlet Allocation (LDA)}
\label{sec:topicmodelling}

Latent Dirichlet Allocation (LDA)\cite{lda} is a generative probabilistic model
designed to uncover latent topics within a collection of discrete data, such as
text corpus.  It represents each document as a mixture of topics, where each
topic is characterized by a distribution over words. In this study, LDA was
applied to the lyrics dataset to identify prevalent themes and topics across
different songs. By representing each song as a distribution over topics, LDA
provided insights into the thematic content of the lyrics, allowing for the
analysis of how these themes correlate with acoustic features and other song
attributes.

In this study, LDA was applied on the lyrics in order to identify commonly
ocurrirng topics in songs. Each song's lyrics were represented as a combination
of topics, and the most representative words for each topic were extracted.
This provided insights into the thematic content of lyrics, enabling further
exploration of relationships between lyrical topics and other features, like
popularity orr acoustic properties.

By combining the topics derived with LDA with additional features, such as
acoustic parameters, the study aimed to analyze the interplay between a song's
musical and lyrical components.


\section{Statistical Methods}
\label{sec:statisticalmethods}

Statistical methods provided a foundation for analyzing complex relationships
between features and a framework for descriptive data analysis and hypothesis
testing.

\subsection{Person Correlation}

Person Corrrelation is a statistical measure used to quantify linear
relationship between two continuous variables. The resulting coefficient ranges
from -1 to 1 and is computed in the following way:

\[
r = \frac{\sum{(x_i - \bar{x})(y_i - \bar{y})}}{\sqrt{\sum{(x_i - \bar{x})^2} \sum{(y_i - \bar{y})^2}}}
\]

Where:
\begin{itemize}
    \item \( x_i \) and \( y_i \): The data points for the two variables.
    \item \( \bar{x} \) and \( \bar{y} \): The mean values of the variables.
\end{itemize}

Interpretation:
\begin{itemize}
  \item An \textit{r} value close to 1 indicates very strong positive linear relationship
  \item An \textit{r} value close to -1 indicates very strong negative linear relationship
  \item An \textit{r} value close to 0 indicates little to no relationship
\end{itemize}



\subsection{Bootstrap Testing}

Bootstrap testing was used to verify hypotheses specified in the research
objectives. It's a resampling-based statistical technique that estimates the
variability of a statistic(e.g. mean or median)  by
repeatedly sampling its values from the dataset with replacement. It's highly
versatile since it doesn't rely on strong distributional assumptions.


\begin{center}
\begin{figure}[ht]
  \centering
  \includegraphics[width=5in]{img/bootstrap.jpg}
  \caption{Illustration of bootstrap resampling: The distribution of the
  statistic (e.g., mean) of the target variable for two different samples
(called groups). This demonstrates the variability of the statistic across
resampled datasets.}
  \label{Figure:fig_beh}
\end{figure}
\end{center}

\chapter{Exploratory Data Analysis}
\label{cha:eda}
%---------------------------------------------------------------------------


\section{Spotify Features}
\label{sec:spotifyfeatures}

\subsection*{Correlation Heatmap}
\label{sec:correlationheatmapsspotifyfeatures}

\begin{center}
\begin{figure}[H]
  \centering
  \includegraphics[width=4in]{img/corr_heatmap_spotify_features.png}
  \caption{Pearson's Correlation Heatmap of Spotify Features}
  \label{Figure:fig_beh}
\end{figure}
\end{center}

\subsection*{Hierarchical Clustering}
\label{sec:hierarchicalclustering}

\begin{center}
\begin{figure}[H]
  \centering
  \includegraphics[width=4in]{img/dendrogram_spotify_features.png}
  \caption{Hierarchical Clustering of Spotify Features}
  \label{Figure:dendrogram_spotify_features}
\end{figure}
\end{center}

\subsection*{Observations}
Spotify audio features show high level of correlation between each other, especially:
\begin{itemize}
  \item \textit{Energy} and \textit{Loudness}: A high correlation(0.74) suggest
    that energetic songs are typically louder.
  \item \textit{Danceability} and \textit{Valence}: high correlation between
    them indicates that songs perceived as positive and happy are usually more
    danceable.
  \item \textit{Energy} and \textit{Acousticness}: there is a strong negative
    correlation between those features, suggesting that high-energy tracks are
    less likely to have acoustic elements.
  \item The dendrogram\ref{Figure:dendrogram_spotify_features} shows relations
    between features and allows us to compare which features correlate with
    each other. In addition to the relationships seen in the correlation
    heatmap, we observe that \textit{popularity} appears to have a weak
    correlation with \textit{release year} and \textit{speechiness}.

\end{itemize}


%---------------------------------------------------------------------------


\section{Lyrical Features}

\subsection*{Correlation Heatmap}
\label{sec:correlationheatmapsspotifyfeatures}

\begin{center}
\begin{figure}[H]
  \centering
  \includegraphics[width=4in]{img/corr_heatmap_lyrical.png}
  \caption{Pearson's Correlation Heatmap of Lyrical Features}
  \label{Figure:fig_beh}
\end{figure}
\end{center}

\subsection*{Hierarchical Clustering}
\label{sec:hierarchicalclustering}

\begin{center}
\begin{figure}[H]
  \centering
  \includegraphics[width=4in]{img/dendrogram_lyrical.png}
  \caption{Hierarchical Clustering of Lyrical Features}
  \label{Figure:dendrogram_spotify_features}
\end{figure}
\end{center}


\subsection*{Observations}

\begin{itemize}
  \item \textit{Flesch Reading Ease}, \textit{Gunning Fog} and \textit{Dale
    Chall} scores exhibit strong positive correlations, highlighting their
    shared focus on measuring lyrical complexity.
  \item \textit{Lexical Richness} correlates with \textit{Word Count},
    indicating that more lexically rrich lyrics tend to have  greater variety
    of words.
  \item On the dendrogram we can distinguish three major clusters of features:
    \begin{itemize}
      \item \textbf{Lyrical Complexity Metrics} - \textit{Flesch Reading Ease},
        \textit{Gunning Fog}, \textit{Dale Chall} and \textit{Syntactic
        Complexity } quantify how difficult and complex the lyrics are.
      \item  \textbf{Lexical Features} - features such as \textit{Type-Token
        Ratio}, \textit{Lexical Richness} and \textit{Unique Word Count} form a
        cohesive cluster.
      \item \textbf{Sentiment-Related Features} - it includes \textit{Sentiment
        Polarity}, \textit{VADER Compound} and \textit{Semantic Depth}, which
        reflect emotional aspects of the lyrics.
    \end{itemize}
  \item Sentiment-related features are relatively closely grouped, suggesting a
    high level of interdependence
\end{itemize}


%---------------------------------------------------------------------------
\section{Audio Features}

\subsection*{Correlation Heatmap}
\label{sec:correlationheatmapsspotifyfeatures}

\begin{center}
\begin{figure}[H]
  \centering
  \includegraphics[width=4in]{img/corr_heatmap_audio.png}
  \caption{Pearson's Correlation Heatmap of Audio Features}
  \label{Figure:fig_beh}
\end{figure}
\end{center}

\subsection*{Hierarchical Clustering}
\label{sec:hierarchicalclustering}

\begin{center}
\begin{figure}[H]
  \centering
  \includegraphics[width=4in]{img/dendrogram_audio.png}
  \caption{Hierarchical Clustering of Audio Features}
  \label{Figure:dendrogram_spotify_features}
\end{figure}
\end{center}


\subsection*{Observations}
\begin{itemize}
  \item \textit{MFCC Features} show strong correlations among themselves(e.g.
    \textit{mfcc\_4}, \textit{mfcc\_6}, \textit{mfcc\_8}, etc.).
  \item Similarily \textit{chroma} features are highly correlated with each
    other, indicating that  they capture similar aspects of tonal energy
    distribution.
  \item \textit{spectral\_contrast} features exhibit relatively small
    correlations with \textit{mfcc} and  \textit{chroma} features, suggesting
    that they capture different characteristics of audio
  \item All \textit{chroma} features are grouped together in the same cluster
    on the dendrogram, supporting the conclusion drawn from the heatmap about
    strong correlations between them
  \item \textit{zcr} and \textit{tempo\_extracted} are relatively independent
  \item A certain degree of overlap between \textit{mfcc} and
    \textit{spectral\_contrast} features is observed in the dendrogram,
    indicating potential shared information in capturing specific audio
    properties.
\end{itemize}

%---------------------------------------------------------------------------


\section{Empath Features}

\subsection*{Correlation Heatmap}
\label{sec:correlationheatmapsspotifyfeatures}

\begin{center}
\begin{figure}[H]
  \centering
  \includegraphics[width=4in]{img/corr_heatmap_empath.png}
  \caption{Pearson's Correlation Heatmap of Empath Features}
  \label{Figure:fig_beh}
\end{figure}
\end{center}

\subsection*{Hierarchical Clustering}
\label{sec:hierarchicalclustering}

\begin{center}
\begin{figure}[H]
  \centering
  \includegraphics[width=4in]{img/dendrogram_empath.png}
  \caption{Hierarchical Clustering of Empath Features}
  \label{Figure:dendrogram_spotify_features}
\end{figure}
\end{center}


\subsection*{Observations}

\begin{itemize}
  \item \textbf{High Correlations Reflect Logical Groupings}: high correlation
    between features such as \textit{empath\_optimism}, \textit{empath\_love} and
    \textit{empath\_affection} align with the intuitive understanding that
    these aspects are closely related.
  \item Observed patters in corrrelations confirm that Empath was designed to
    group semantically related concepts together.
  \item The visualizations align with the intended design of Empath as a tool
    for interpretable feature  extraction.
\end{itemize}


%---------------------------------------------------------------------------

\section{Genre Analysis}

\subsection{Genre Similarity}

\begin{center}
\begin{figure}[H]
  \centering
  \includegraphics[width=5in]{img/genres_dendrogram.png}
  \caption{Hierarchical Clustering of Genres by lyrical and audio features}
  \label{Figure:dendrogram_spotify_features}
\end{figure}
\end{center}

\begin{center}
\begin{figure}[H]
  \centering
  \includegraphics[width=4in]{img/genres_similarity_heatmap.png}
  \caption{Heatmap of euclidean distance of Genres calculated on lyrical and audio features}
  \label{Figure:dendrogram_spotify_features}
\end{figure}
\end{center}

The dendrogram illustrates hierarchical clustering of genres based on their
lyrical and  audio features. The clustering aligns with general cultural and
musical understandings of  these genres. For instance, closely related  genres
like hip hop and rap are grouped  together, indicating that they're highly
similar.

The heatmap further supports this observation. The Euclidean distances of
\textit{rap} and \textit{hip hop} with other genres stand  out as the highest.

Interestingly, genres such as \textit{metal} and \textit{indie}, while distinct
in sound, share some overlap in features, which is reflected on the dendrogram.

\begin{center}
\begin{figure}[H]
  \centering
  \includegraphics[width=6in]{img/heatmap_of_empath.png}
  \caption{Heatmap of mean values of empath features in each genre}
  \label{Figure:dendrogram_spotify_features}
\end{figure}
\end{center}

The heatmap displays scaled mean values of \textit{Empath features} for
different musical genres, highlighting thematic  differences in their lyrics.
It can be observed that:
\begin{itemize}
  \item \textbf{Country} music lyrics often involve topics related to family,
    love and  positive emotions. It seems to align well with genre's tendency
    to narrate personal, heartfelt and often nostalgic stories.
  \item \textbf{EDM} lyrics seem to often touch upon love and optimism.
  \item For both \textbf{Hip Hop} and \textbf{Rap} according to Empath  the
    most prominent topic is \textit{speaking}.
  \item For \textbf{Metal} High values in \textit{negative emotion }and
    \textit{pain} underscore the genre's focus on intense, darker themes.
  \item Elevated values in \textit{love} and \textit{positive emotion}
    highlight \textbf{Pop} music’s focus on themes of love and relationships.
  \item High values in \textit{friends } and \textit{optimism} categories
    reflect \textbf{Reggae's} emphasis on positivity, social connection, and
    uplifting messages.
\end{itemize}

\subsection{Lyrical Similarity Based on Embeddings}
\begin{center}
\begin{figure}[H]
  \centering
  \includegraphics[width=6in]{img/tsne_genres.png}
  \caption{Heatmap highlighting the maximum feature values across genres.}
  \label{Figure:dendrogram_spotify_features}
\end{figure}
\end{center}

\textit{t-SNE (t-Distributed Stochastic Neighbor Embedding)} was applied to the
Euclidean distance matrix of mean standardized Word2Vec and TF-IDF embeddings
for each genre. This dimensionality reduction technique allows to project
high-dimensional data into a 2D space while preserving the relative distances
and similarities between data points as much as possible. This enables to
visualize lyrical similarities between different genres.

In order to further explore lyrical relationships between the genres K-means
clustering was applied on the mean standardized embeddings per genre,The
optimal number of clusters was determined using the elbow method, which
suggested four distinct clusters. This additional operation clusters the most
similar genres together.

\begin{itemize}
  \item \textit{Hip hop} and \textit{rap} form a distinct cluster, indicating
    strong lyrical and thematic similarity.
  \item \textit{Reggae} is slightly separate from other genres but still part
    of a larger cluster that includes \textit{rock}, \textit{pop},
    \textit{country}, and \textit{R\&B}, suggesting moderate similarity in
    lyrical and thematic content of songs belonging to those genres.
  \item \textit{Metal} and \textit{indie} are closely positioned, sharing
    overlapping themes while forming a separate cluster.
  \item \textit{EDM} is positioned furthest from other genres, highlighting
    its unique lyrical and thematic style.
\end{itemize}

\subsection{Top Genre Characteristics}
\begin{center}
\begin{figure}[H]
  \centering
  \includegraphics[width=6in]{img/heatmap_max_feature_values_by_genre.png}
  \caption{Heatmap highlighting the maximum feature values across genres.}
  \label{Figure:dendrogram_spotify_features}
\end{figure}
\end{center}
Each filled cell represents the genre that exhibits the highest mean value for
the corresponding feature, calculated using scaled feature values. This
visualization emphasizes distinctive characteristics of each genre, and allows
to find out ``in  which genre the feature achieved its highest mean``.

\begin{itemize}
  \item \textbf{Country}: Highest in acousticness and rhyme density
  \item \textbf{EDM}: Dominated in tempo, loudness, and featured the most
    recent songs
  \item \textbf{Hip Hop}: Excelled in unique word count, lexical richness, and
    semantic depth.
  \item \textbf{Metal}: Stood out for energy and longer song durations.
  \item \textbf{Pop}: Showed the highest positivity (VADER compound) and
    subjectivity in lyrics.
  \item \textbf{Rap}: Highest in repetition count and reading ease.
  \item \textbf{Reggae}: Highlighted by high valence and danceability.
  \item \textbf{Rock}: Scored highest in in popularity.
    
\end{itemize}

\begin{center}
\begin{figure}[H]
  \centering
  \includegraphics[width=6in]{img/heatmap_top_feature_values_by_genre.png}
  \caption{Heatmap highlighting the top 5 features by genre. Each filled cell
  represents one of the five highest mean feature values for a given genre,
calculated on scaled data. This visualization focuses on identifying the most
distinctive traits for each genre.}
  \label{Figure:dendrogram_spotify_features}
\end{figure}
\end{center}

Lastly, we switch perspectives, focusing not on individual features but instead
examining each genrer to identify the top 5 features that characterize it. Each
filled cell in that heatmap shows one of the five most prominent
characteristics exhibited by that genre.


\chapter{Discussion}
\label{cha:discussion}
%---------------------------------------------------------------------------

\section{Key Findings}
\label{sec:keyfindings}


%---------------------------------------------------------------------------

\section{Limitations of the Study}
\label{sec:limitationsofthestudy}


\chapter{Conclusion}
\label{cha:conclusion}
%---------------------------------------------------------------------------

\section{Summary of Contributions}
\label{sec:summaryofcontributions}

This thesis has provided an extensive analysis of the relationships between
acoustic parameters, lyrical content, and metadata in characterizing music
tracks. By integrating well organized data collection methodology, advanced
feature engineering, explainable AI methods, machine learning models and
statistical methods, this study accomplished the following:

\begin{itemize}
  \item Constructed a diverse dataset by combining Spotify metadata and audio
    features, textual features derived from lyrics, and acoustic features
    extracted from downloaded MP3 files, representing over 3,500 songs;

  \item Employed advanced statistical methods to conduct extensive exploratory
    data analysis to identify relationships, patterns, and distinctive
    characteristics within acoustic and lyrical features across genres;

  \item Found distinctive traits of the genres included in the study,
    describing their unique acoustic and lyrical attributes;

  \item Trained machine learning models for various predictive tasks, including
    regression of popularity and  classification of explicitness, sentiment,
    and genre. and identified key features influencing these attributes. The
    models were tested against baseline models to evaluate their performance,
    and explainable AI methodologies such as SHAP were used to enhance
    interpretability and investigate feature importance and interactions;

  \item Applied Latent Dirichlet Allocation (LDA) to identify and analyze
    thematic patterns in lyrics, providing insights into the recurring themes
    that can be found in the songs included in the dataset as well as their
    distribution across genres and decades, and their distinct characteristics;

  \item With the use of statistical methods it investigated temporal trends in
    music characteristics, identifying changes reflecting the evolution of
    music styles and listener preferences across the decades;

  \item Created a reusable framework for data collection, feature engineering
    and XAI-driven analysis for music, providing foundation for further
    research in similar domains.
\end{itemize}






%---------------------------------------------------------------------------
\section{Limitations of the Study}
\label{sec:limitations}

When interpreting the results, several limitations of the  study should be
considered:

\subsubsection*{Selection Bias}

In order to acquire the data in the data collection phase the method chosen for
sampling the songs available on Spotify utilized  stratified sampling based on
genre and release decade. This method of sampling relied on Spotify's search
algorithm, and therefore introduced a potential bias influenced by factors that
this thesis did not account for. Factors like the musical preferences of the
user querying the data, the time of the year at which the data was acquired,
and many other features influencing the search algorithm's results likely
impacted the selection of songs, potentially skewing the dataset. Collection of
truly random sample for the purpose of this study would have posed significant
challenges, explained in \textit{Sampling Method} (\ref{sec:samplingmethod}).


\subsubsection*{Dimensionality}

Due to computational and memory constraints the dataset was relatively small.
A larger sample would most likely provide better quality insights and improve
the performance of the predictive models, leading to more robust findings. 


\subsubsection*{Dependency on Spotify}
Many features used in this study were derived from Spotify's metadata and
algorithms. Genre labels, popularity metrics, and features such as danceability
and loudness depend on Spotify's estimations, which may not always align with
objective or universal understanding of those properties.



%---------------------------------------------------------------------------
\section{Recommendations for Future Work}
\label{sec:recommendationsforfuturework}


\subsubsection*{Gather More Data and Increase Diversity}
Collecting more data with possibly a different sampling method and adding
more diverse features would provide a better foundation for research. It would
not only improve the quality of trained predictive models but also contribute
to more comprehensive and accurate analyses, uncovering new findings or patterns.

\subsubsection*{Reduce Dependency on Spotify}
Incorporating alternative sources into the data collection proccess would reduce
the reliance on Spotify's metadata and features generated using its models.
Possible options would be publicly available audio datasets, data from other
streaming platforms, or more extensive feature extraction from raw audio files.


\subsubsection*{Improve Feature Engineering Methodology}
The feature  engineering process could be greatly enhanced, with particular
focus on more advanced extraction of features from MP3 files. Deeper analysis
of importance of these features and more advanced feature selection would be
necessary. Improved feature engineering would give better insight into music's
structure and allow for exploration for more relationships and dependencies
with lyrical features, as well as contribute to more accurate analyses and
better performing models.

\subsubsection*{Implementation and Evaluation of More Advanced Modeling Techniques}
The study focused on model interpretability and for simplicity focused only on
using Catboost, since the goal was getting insight into model's predictive
processes and understanding relationships between features. More complex models
and architectures could likely strongly outperform the models trained in this
thesis and yield much more accurate results with the same set of features. Due
to computational and time constraints the searches with Optuna were also
relatively shallow, but the parameter grids could be much larger and numbers of
trials much higher with enough resources, allowing to find the best model
configurations.

\subsubsection*{Improve Clustering and Topic Modeling}
The clustering and topic modelling techniques could be largely extended and
provide much more information with enough effort. The topics extracted with LDA
could also serve as features in training of next models.

% \include{rozdzial8}
% \include{rozdzial9}

\cleardoublepage % Dodaj nową stronę, aby "Literatura" zaczęła się na nieparzystej stronie
\phantomsection % Dodaj punkt w spisie treści dla sekcji "Literatura"
\addcontentsline{toc}{chapter}{Bibliografia} % Dodaj "Literatura" do spisu treści, bez numerowania
\printbibliography[title={Bibliografia}]
\end{document}
