\chapter{Data}
\label{cha:data}
%---------------------------------------------------------------------------

\section{Dataset Description}
\label{sec:datasetdescription}

The dataset consists of around 6000 songs. For each song metadata, audio
recording, lyrics and spotify audio features were fetched.

\section{Data Collection Methods}
\label{sec:datacollectionmethods}

\subsection{Sources}
The data was collected from various sources. Metadata and audio features
were fetched from Spotify API. The lyrics were scraped from MusixMatch,
LetrasMus, MakeItPersonal or Lyrics Fandom, or fetched via the Genius API,
depending on  the availability. Audio files were downloaded from YouTube and
saved as mp3 files.

\subsection{Methodology}
The data collection process was automated for efficiency. The starting point
was a list of Spotify playlist URIs. Each playlist was processed sequentially. 
The script fetched metadata and audio features for each song in the
playlist.

Lyrics were then searched based on \textit{artist name} and \textit{title} of
each song, using multiple lyrics providers. The system continued to query
different sources until it found lyrics in at least one of them. If it failed
to retrieve lyrics for the song, it was discarded and wouldn't make it to the
final dataset.

Finally, the script  searched YouTube for each song and downloaded the first
relevant result, saving it to an mp3 file. All data was saved into a CSV
file named after the playlist and stored alongside the recordings.

The entire process was parallelized to significantly enhance the speed of data
acquisition. It was designed in a robust manner, with careful error handling
and ability to stop the process at any  time and pick up where it left off.

Additionally, to optimize performance, the entire process was parallelized,
significantly increasing the speed of data acquisition. It was designed with
robust error handling, ensuring data correctness and completeness. Moreover, it
featured the ability to pause and seamlessly resume the process, continuing
exactly where it left off without data loss.


%---------------------------------------------------------------------------


\subsection{Metadata}
\label{sec:metadata}
Song's information downloaded from Spotify API. The information it includes is:

\begin{itemize}
  \item \textbf{Popularity} - relative measure with values ranging from 0 to
    100 describing how popular the song is, estimated mostly based on total
    number of plays and how recent  those plays are
  \item \textbf{Explicitness} - whether or not the song contains explicit lyrics
  \item \textbf{Genre} - main genre of the artist(Spotify does not provide
    information about genre of each musical track)
  \item \textbf{Album Release Year} - the release year of the album that the
    song originates from
\end{itemize}

\textit{popularity}, \textit{explicitness},
\textit{genre}, and \textit{release year}. Since Spotify does not provide
information about the genre of specific songs, the genre extracted is the main
genre of the primary artist in the song.


\subsection{Spotify Audio Features}
\label{sec:spotifyaudiofeatures}
Those features were fetched from the Spotify API.They describe different
acoustic properties songs:
\begin{itemize}
  \item \textbf{Speechiness} - relative measure of spoken words in a track
  \item \textbf{Acousticness} - a confidence measure of whether the song is
    acoustic
  \item \textbf{Danceability} - a measure of how suitable the song is for
    dancing. Its based on parameters like tempo, rhythm stability, beat
    strength etc.
  \item \textbf{Energy} - a measure of perceived intensity of songs. Energetic
    tracks are usually louder, faster, feel more intense.
  \item \textbf{Loudness} - overall loudness of the  track in dB averaged
    across the entire track
  \item \textbf{Valence} - relative measure describing musical positiveness of
    a track
  \item \textbf{Instrumentalness} - measure of how likelihood of the track not
    containing vocals. In this paper since lyrics are mandatory it's used to
    discard instrumental tracks.
  \item \textbf{Liveness} - probability of the song being recorded during a
    live performance.
  \item \textbf{Key} - the key of the song, e.g. C\#.
  \item \textbf{Mode} - indicates the modality of a track(major / minor)
  \item \textbf{Tempo} - estimated tempo of a track in beats per minute(BPM)
  \item \textbf{Time Signature} - specifies how many beats there are in each
    bar
  \item \textbf{Duration} - the duration of the track in milliseconds
\end{itemize}


\subsection{Lyrics Features}
 The lyrics serve as a textual representation of the song's thematic,
 emotional, and linguistic elements. Since they came from various data sources,
 they had  to undergo cleaning procedure in order to remove faulty information
 and prepare them to be processed by the textual feature extraction class. The
 process consisted of:
 \begin{itemize}
  \item \textbf{Standardization} - lyrics were converted to lowercase
  \item \textbf{Noise Removal} - unnecessary characters, numbers and
    punctuation, as well as additional comments used by lyrics providers(e.g.
    'chorus') were removed
  \item \textbf{Stopwords Filtration} - exclusion of frequently occuring words
    that carry little information, like 'the' in English
  \item \textbf{Stemming} - words were reduced to their root forms to enhance
    uniformity and reduce corpus size
 \end{itemize}

 This process laid foundation for further extraction of textual features used
 for exploratory data analysis, statistical inference and training ML models.
 That process will be explained in later chapter.


%---------------------------------------------------------------------------

\section{Tools and Libraries Used}
\label{sec:toolsandlibrariesused}
Python libraries used to facilitate the data acquisition were:
\begin{itemize}
  \item \textit{Spotipy} - a lightweight python library for Spotify API
  \item \textit{youtube-dl} - a library used to find and download youtube videos
  \item \textit{BeautifulSoup} - a library used for extracting information from
    HTML, commonly used for web scraping
\end{itemize}



