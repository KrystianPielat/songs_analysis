\chapter{Data}
\label{cha:data}


%---------------------------------------------------------------------------

\section{Sampling Method}
\label{sec:samplingmethod}


Spotify’s catalog contains approximately 100 million songs and over 6000
distinct genres. Collecting a truly random sample from this huge and diverse
population presents significant challenges. A purely random sampling approach
would likely result in a dataset heavily skewed towards obscure genres and
imbalanced release year distribution. This would lead to poor data quality for
meeting the research objectives, with sparse meaningful relationships and an
overwhelming diversity of genres. Such diversity would make it nearly
impossible to extract actionable insights and relationships, as less
represented genres and rare tracks would dominate, shifting the focus from more
prominent trends and patterns in music.


To address these challenges, a smaller and more organized sample of 3500 songs
was selected. The stratified sampling approach was designed to ensure a
balanced dataset by focusing on two key factors: genre and release year. The
sampling process utilized Spotify's query parametrization tool, which allows
for the specification of desired genres and release year ranges for musical
tracks. Release years were grouped into 10-year intervals, starting from 1950
and extending to 2020, ensuring the dataset represents different time periods
in modern music. The selection of genres was arbitrary and aimed to include a
mix of popular and diverse styles, providing a robust representation of
mainstream musical styles.

Additionally, in order to enable more precise and insightful topics extraction
and uniformity of the data, all songs whose lyrics weren't identified to be in
English were discarded.
 
While this approach introduces some degree of selection bias, favoring tracks
that Spotify's search algorithm prioritizes, this bias is acceptable given the
research objectives. By explicitly focusing on popular genres and balancing
across release years, the sample aims to capture relationships and patterns
that are representative of mainstream trends and music evolution. The
structured stratification ensures that each genre and time period is adequately
represented, providing a comprehensive and interpretable dataset for  the
analysis.


%---------------------------------------------------------------------------

\section{Dataset Description}
\label{sec:datasetdescription}

The dataset consists of around 3500 songs in English. For each song metadata, audio
recording, lyrics and spotify audio features were fetched.

\section{Data Collection Methods}
\label{sec:datacollectionmethods}

\subsection{Sources}
The data was collected from various sources. Metadata and audio features were
fetched from Spotify API\cite{spotify_api}. The lyrics were scraped from
LetrasMus\cite{letras_mus}, MakeItPersonal\cite{makeitpersonal} or Lyrics
Fandom\cite{lyricsfandom}, or fetched via the Genius API\cite{genius},
depending on the availability. Audio files were downloaded from YouTube using
youtube-dl\cite{ytdl} library and saved as mp3 files.

\subsection{Methodology}
The data collection process was automated for efficiency. The starting point
was a list of genres and decades. The script leveraged Spotify API's search
query parametrization to fetch equal number of songs from each genre and in
released in  each decade. For every song its metadata and audio featurers were
fetched.

Lyrics were then searched based on \textit{artist name} and \textit{title} of
each song, using multiple lyrics providers. The system continued to query
different sources until it found lyrics in at least one of them. If it failed
to retrieve lyrics for the song, it was discarded and wouldn't make it to the
final dataset.

Finally, the script  searched YouTube for each song and downloaded the first
relevant result, saving it to an mp3 file. All data was saved into a CSV
file named after the playlist and stored alongside the recordings.

The entire process was parallelized to significantly enhance the speed of data
acquisition. It was designed in a robust manner, with careful error handling
and ability to stop the process at any  time and pick up where it left off.


%---------------------------------------------------------------------------


\subsection{Metadata}
\label{sec:metadata}
Metadata included song's information downloaded from Spotify API, precisely:

\begin{itemize}
  \item \textbf{Popularity} - relative measure with values ranging from 0 to
    100 describing how popular the song is, estimated mostly based on total
    number of plays and how recent  those plays are.
  \item \textbf{Explicitness} - whether or not the song contains explicit
    lyrics.
  \item \textbf{Genre} - the genre specified in the search query that returned
    this song.
  \item \textbf{Album Release Year} - the release year of the album that the
    song originates from.
\end{itemize}


\subsection{Spotify Audio Features}
\label{sec:spotifyaudiofeatures}
Those features were fetched from the Spotify API.They describe different
acoustic properties of the songs:
\begin{itemize}
  \item \textbf{Speechiness} - relative measure of spoken words in a track.
  \item \textbf{Acousticness} - a confidence measure of whether the song is
    acoustic.
  \item \textbf{Danceability} - a measure of how suitable the song is for
    dancing. Its based on parameters like tempo, rhythm stability, beat
    strength etc.
  \item \textbf{Energy} - a measure of perceived intensity of songs. Energetic
    tracks are usually louder, faster, feel more intense.
  \item \textbf{Loudness} - overall loudness of the  track in dB averaged
    across the entire track.
  \item \textbf{Valence} - relative measure describing musical positiveness of
    a track.
  \item \textbf{Instrumentalness} - likelihood of the track not containing
    vocals. In this paper since lyrics are mandatory it's used to discard
    instrumental tracks.
  \item \textbf{Liveness} - probability of the song being recorded during a
    live performance.
  \item \textbf{Key} - the key of the song, e.g. C\#.
  \item \textbf{Mode} - indicates the modality of a track(major / minor).
  \item \textbf{Tempo} - estimated tempo of a track in beats per minute(BPM).
  \item \textbf{Time Signature} - specifies how many beats there are in each
    bar.
  \item \textbf{Duration} - the duration of the track in milliseconds.
\end{itemize}


\subsection{Lyrics Features}
 The lyrics serve as a textual representation of the song's thematic,
 emotional, and linguistic elements. Since they came from various data sources,
 they had  to undergo cleaning procedure in order to remove faulty information
 and prepare them to be processed by the text features extraction class. The
 process consisted of:
 \begin{itemize}
  \item \textbf{Standardization} - lyrics were converted to lowercase.
  \item \textbf{Noise Removal} - unnecessary characters, numbers and
    punctuation, as well as additional comments used by lyrics providers(e.g.
    'chorus') were removed.
  \item \textbf{Stopwords Filtration} - exclusion of frequently occuring words
    that carry little information, like 'the' in English.
  \item \textbf{Stemming} - words were reduced to their root forms to enhance
    uniformity and reduce corpus size.
 \end{itemize}

 This process laid foundation for further extraction of textual features used
 for exploratory data analysis, statistical inference and training ML models.
 That process will be explained in later chapter.


%---------------------------------------------------------------------------

\section{Tools and Libraries Used}
\label{sec:toolsandlibrariesused}
Python libraries used to for data acquisition were:
\begin{itemize}
  \item \textit{Spotipy} \cite{spotipy}- a lightweight python library for
    Spotify API.
  \item \textit{youtube-dl} \cite{ytdl} - a library used to find and download
    YouTube videos.
  \item \textit{BeautifulSoup} \cite{beautifulsoup} - a library used for
    extracting information from HTML, commonly used for web scraping.
\end{itemize}



