\chapter{Related Work}
\label{cha:literaturereview}

%---------------------------------------------------------------------------

\section{Acoustic Features in Music Analysis}
\label{sec:acousticfeaturesinmusicanalysis}

\subsubsection*{``Music Genre Classification Using MFCC, k-NN, and SVM
Classifier``}


This study explores music genre classification using MFCCs and Chroma features
on the GTZAN dataset, that consists of 900 tracks across 9 genres. The
best-performing model was an SVM with a polynomial kernel, achieving accuracy
of 78\%. Some genres were identified to have overlapping characteristics which
posed problems for the classification model.\cite{music_genre_classification_mfcc}

The paper highlights the effectiveness of MFCCs and Chroma features for
audio-based classification, aligning with this thesis's use of acoustic
features. However, unlike this study, the thesis extends the analysis by
incorporating lyrical features, metadata, and audio features provided by
Spotify, addressing limitations in feature diversity.

\subsubsection*{``Classifying Music Audio with Timbral and Chroma Features``}

The paper ``Classifying Music Audio with Timbral and Chroma Features`` examines
the use of timbral (e.g., MFCCs) and harmonic (e.g., chroma) features in music
classification tasks. The study highlights that MFCCs, which represent timbral
qualities, are effective for tasks like \textbf{artist identification, achieving 56\%
accuracy in a 20-way classification task}. When combined with beat-synchronous
chroma features, which capture harmonic and melodic information, the model's
accuracy improved to 59\%. This demonstrates the importance of leveraging
complementary audio features for more robust classification. These findings
emphasize the potential of combining diverse acoustic features in predictive
models, aligning closely with the objectives of this thesis.\cite{classifying_music_audio}



%---------------------------------------------------------------------------

\section{Lyrical Features in Music Analysis}
\label{sec:lyricalfeaturesinmusicanalysis}

\subsubsection*{``Using Machine Learning Analysis to Interpret the Relationship
Between Music Emotion and Lyric Features``}


This study investigates the relationship between lyrical features and perceived
music emotions using 2,372 Chinese pop songs. Lyric features were extracted
with LIWC (Linguistic Inquiry and Word Count), and audio features
such as MFCCs and chroma were derived using Librosa. The analysis highlighted
that lyrical features like the frequency of positive and negative emotion words
contributed significantly to predicting the perceived valence of music, whereas
audio features were dominant in predicting perceived arousal.\cite{valence_and_lyrics}

The study utilized Random Forest regression models, demonstrating that
combining lyric and audio features improved \textbf{predictions of valence,
with an $R^2$ of 0.481}, but lyrics had little impact on arousal models. These
findings align with this thesis's use of both lyrical and acoustic features for
predictive tasks but differ in the application of explainable AI (XAI)
techniques like SHAP for feature interpretation. Moreover, this thesis expands
this study by incorporating Spotify audio features and metadata for broader
analytical capabilities and more diverse research objectives.


\subsubsection*{``Sentiment Analysis and Lyrics Theme Recognition Using NLP Techniques``}


This paper investigates the relationship between sentiment and themes in music
lyrics using Natural Language Processing (NLP) techniques. It applies sentiment
analysis and thematic recognition across a diverse dataset, identifying
emotional nuances and recurring topics in the lyrics. The analysis identifies
correlations between sentiment categories (positive, negative, neutral) and
thematic clusters (e.g., love, social justice, personal reflection). The
authors use techniques like Latent Dirichlet Allocation (LDA) for topic
modeling and Support Vector Machines (SVM) for sentiment classification.\cite{du_2024}

In the context of this thesis, this study aligns closely with the focus on
lyrical analysis, particularly in employing NLP-driven sentiment and thematic
classification. However, while this work emphasizes standalone lyric-based
analysis, this thesis extends the methodology by integrating Spotify’s audio
features, acoustic analysis, and metadata. 

%---------------------------------------------------------------------------

\section{Machine Learning in Music Analysis}
\label{sec:machinelearningfeaturesinmusicanalysis}

\subsubsection*{``Beyond Beats: A Recipe to Song Popularity? A Machine Learning Approach``}

The paper ``Beyond Beats`` investigates the predictive power of various machine
learning models for song popularity on a dataset of 30,000 songs spanning six
genres. It focuses on metadata and audio features fetched from
Spotify(\textit{danceability}, \textit{acousticness} etc.), as well as the
genre. The best performing model was \textbf{Random Forest and achieved 16.31 MAE.}
Predictions across all methods remained relatively modest, reflecting the
complex and multi-dimensional  nature of song popularity. Those results can be
greatly improved using additional features. The authors noted that predictive
accuracy was constrained by the absence of post-release factors such as
marketing, social media reception, and artist reputation.\cite{beyond_beats}


\subsubsection*{``Predicting Song Popularity in the Digital Age Through
Spotify’s Data``}  
Similarily to the previous one, the paper ``Predicting Song Popularity in the
Digital Age Through Spotify’s Data`` explores the relationship between
Spotify's audio features  and song popularity, using a dataset spanning from
1986 to 2022. The study employed linear regression to predict popularity and
\textbf{achieved an adjusted $R^2$ of 0.38}, highlighting the moderate
predictive power of these features. The analysis revealed that attributes such
as \textit{danceability} and \textit{duration} positively correlate with
popularity, while \textit{speechiness} tends to have a negative
impact.\cite{predicting_song_popularity_2024}
